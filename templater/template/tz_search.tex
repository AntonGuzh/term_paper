% Это основная команда, с которой начинается любой \LaTeX-файл. Она отвечает за тип документа, с которым связаны основные правил оформления текста.
\documentclass[allcolors=black]{article}

% Здесь идет преамбула документа, тут пишутся команды, которые настраивают LaTeX окружение, подключаете внешние пакеты, определяете свои команды и окружения. В данном случае я это делаю в отдельных файлах, а тут подключаю эти файлы.

% Здесь я подключаю разные стилевые пакеты. Например возможности набирать особые символы или возможность компилировать русский текст. Подробное описание внутри.
\usepackage{tex/packages}

% Здесь я определяю разные окружения, например, теоремы, определения, замечания и так далее. У этих окружений разные стили оформления, кроме того, эти окружения могут быть нумерованными или нет. Все подробно объяснено внутри.
\usepackage{tex/environments}

% Здесь я определяю разные команды, которых нет в LaTeX, но мне нужны, например, команда \tr для обозначения следа матрицы. Или я переопределяю LaTeX команды, которые работают не так, как мне хотелось бы. Типичный пример мнимая и вещественная часть комплексного числа \Im, \Re. В оригинале они выглядят не так, как мы привыкли. Кроме того, \Im еще используется и для обозначения образа линейного отображения. Подробнее описано внутри.
\usepackage{tex/commands}

% Потребуется для вставки картинки подписи
% \usepackage{graphicx}

% Пакет для титульника проекта
\usepackage{tex/titlepage}

% Здесь задаем параметры титульной страницы
%\setUDK{192.168.1.1}
% Выбрать одно из двух
\setToResearch
%\setToProgram

\setTitle{\VAR{model.user.title}}

% Выбрать одно из трех:
% КТ1 -- \setStageOne
% КТ2 -- \setStageTwo
% Финальная версия -- \setStageFinal
\setStageOne
%\setStageTwo
%\setStageFinal

\setGroup{\VAR{model.user.group}}
\setStudent{\VAR{model.user.student}}
\setStudentDate{\VAR{model.user.date.strftime('%d.%m.%Y')}}
\setAdvisor{\VAR{model.user.advisor}}
\setAdvisorTitle{\VAR{model.user.advisor_title}}
\setAdvisorAffiliation{\VAR{model.user.advisor_affiliation}}
\setAdvisorDate{}
\setGrade{}
\setYear{\VAR{model.user.year}}


% С этого момента начинается текст документа
\begin{document}

% Эта команда создает титульную страницу
\makeTitlePage

% Здесь будет автоматически генерироваться содержание документа
\tableofcontents

\section{Основные термины и определения}
\begin{itemize}
    \BLOCK{ for item in model.terms: }
    \item \textit{\VAR{item.name}} --- \VAR{item.text}
    \BLOCK{ endfor }
\end{itemize}

\section{Введение}
\VAR{model.intro}

\section{Обзор и сравнительный анализ источников}

\BLOCK{ for item in model.sources_review: }
\subsubsection*{\VAR{item.name}}
\VAR{item.text}
\BLOCK{ endfor }

\subsection*{Итоги}
\VAR{model.sources_summary}

\section{Промежуточные результаты}
\BLOCK{for item in model.intermediate_results:}
\VAR{item}

\BLOCK{ endfor }


\section{Календарный план выполнения проекта}
\begin{tabular}{|p{12cm}|p{3cm}|}\hline
    \textbf{Событие} & \textbf{Дедлайн} \\\hline
    \BLOCK{for item in model.plan:}
    \VAR{item.action} & \VAR{item.date.strftime('%d.%m.%Y')}\\\hline
    \BLOCK{ endfor }
\end{tabular}


\end{document}
