% Это основная команда, с которой начинается любой \LaTeX-файл. Она отвечает за тип документа, с которым связаны основные правил оформления текста.
\documentclass[allcolors=black]{article}

% Здесь идет преамбула документа, тут пишутся команды, которые настраивают LaTeX окружение, подключаете внешние пакеты, определяете свои команды и окружения. В данном случае я это делаю в отдельных файлах, а тут подключаю эти файлы.

% Здесь я подключаю разные стилевые пакеты. Например возможности набирать особые символы или возможность компилировать русский текст. Подробное описание внутри.
\usepackage{tex/packages}

% Здесь я определяю разные окружения, например, теоремы, определения, замечания и так далее. У этих окружений разные стили оформления, кроме того, эти окружения могут быть нумерованными или нет. Все подробно объяснено внутри.
\usepackage{tex/environments}

% Здесь я определяю разные команды, которых нет в LaTeX, но мне нужны, например, команда \tr для обозначения следа матрицы. Или я переопределяю LaTeX команды, которые работают не так, как мне хотелось бы. Типичный пример мнимая и вещественная часть комплексного числа \Im, \Re. В оригинале они выглядят не так, как мы привыкли. Кроме того, \Im еще используется и для обозначения образа линейного отображения. Подробнее описано внутри.
\usepackage{tex/commands}

% Потребуется для вставки картинки подписи
% \usepackage{graphicx}

% Пакет для титульника проекта
\usepackage{tex/titlepage}

% Здесь задаем параметры титульной страницы
%\setUDK{192.168.1.1}
% Выбрать одно из двух
%\setToResearch
\setToPmi

\setTitle{\VAR{model.user.title}}

\setGroup{\VAR{model.user.group}}
\setStudent{\VAR{model.user.student}}
\setStudentDate{\VAR{model.user.date.strftime('%d.%m.%Y')}}
\setAdvisor{\VAR{model.user.advisor}}
\setAdvisorTitle{\VAR{model.user.advisor_title}}
\setAdvisorAffiliation{\VAR{model.user.advisor_affiliation}}
\setAdvisorDate{}
\setGrade{}
\setYear{\VAR{model.user.year}}


% С этого момента начинается текст документа
\begin{document}

% Эта команда создает титульную страницу
\makeTitlePage

% Здесь будет автоматически генерироваться содержание документа
\tableofcontents

    \section{Объект испытаний}
    \subsection{Наименование испытуемой программы}
    Наименовыние --- <<\VAR{model.user.title}>>
    
    \subsection{Область применения испытуемой программы}
    \VAR{model.space}
    
    \section{Цель испытаний}
    \VAR{model.goal}
    
    \section{Требования к программе}
    \begin{enumerate}
        \BLOCK{ for item in model.prog_req: }
        \item \VAR{item}
        \BLOCK{ endfor }
    \end{enumerate}


    \section{Требования к программной документации, предъявляемой на испытания}
    \begin{enumerate}
        \BLOCK{ for item in model.doc_req: }
        \item \VAR{item}
        \BLOCK{ endfor }
    \end{enumerate}

    \section{Средства и порядок испытаний}
    \subsection{Технические средства, используемые во время испытаний}
    \begin{enumerate}
        \BLOCK{ for item in model.pc_source: }
        \item \VAR{item}
        \BLOCK{ endfor }
    \end{enumerate}

    \subsection{Программные средства, используемые во время испытаний}
    \begin{enumerate}
        \BLOCK{ for item in model.prog_source: }
        \item \VAR{item}
        \BLOCK{ endfor }
    \end{enumerate}
    
    \subsection{Порядок проведения испытаний}
    \begin{itemize}
        \BLOCK{ for i in range(model.steps|length): }
        \item \VAR{i + 1} этап --- \VAR{model.steps[i].name}
        \BLOCK{ endfor }
    \end{itemize}

    \BLOCK{ for item in model.steps: }
    \subsubsection{\VAR{item.name}}
    \BLOCK{ for line in item.items: }
    \VAR{line}

    \BLOCK{ endfor }
    \BLOCK{ endfor }

    \subsection{Количественные и качественные характеристики, подлежащие оценке}
    \subsubsection{Количественные характеристики, подлежащие оценке}
    \begin{itemize}
        \BLOCK{ for item in model.quantity: }
        \item \textit{\VAR{item.name}} --- \VAR{item.text}
        \BLOCK{ endfor }
    \end{itemize}

    \subsubsection{Качественные характеристики, подлежащие оценке}
    \begin{itemize}
        \BLOCK{ for item in model.quality: }
        \item \textit{\VAR{item.name}} --- \VAR{item.text}
        \BLOCK{ endfor }
    \end{itemize}

    \subsection{Условия проведения испытаний}
    \BLOCK{ for item in model.conditions: }
    \subsubsection{\VAR{item.name}}
    \BLOCK{ for line in item.items: }
    \VAR{line}

    \BLOCK{ endfor }
    \BLOCK{ endfor }

    \section{Методы проведения испытаний }
    \BLOCK{ for item in model.methods: }
    \subsection{\VAR{item.name}}
    \BLOCK{ for line in item.items: }
    \VAR{line}

    \BLOCK{ endfor }
    \BLOCK{ endfor }

\end{document}
