% Это основная команда, с которой начинается любой \LaTeX-файл. Она отвечает за тип документа, с которым связаны основные правил оформления текста.
\documentclass[allcolors=black]{article}

% Здесь идет преамбула документа, тут пишутся команды, которые настраивают LaTeX окружение, подключаете внешние пакеты, определяете свои команды и окружения. В данном случае я это делаю в отдельных файлах, а тут подключаю эти файлы.

% Здесь я подключаю разные стилевые пакеты. Например возможности набирать особые символы или возможность компилировать русский текст. Подробное описание внутри.
\usepackage{tex/packages}

% Здесь я определяю разные окружения, например, теоремы, определения, замечания и так далее. У этих окружений разные стили оформления, кроме того, эти окружения могут быть нумерованными или нет. Все подробно объяснено внутри.
\usepackage{tex/environments}

% Здесь я определяю разные команды, которых нет в LaTeX, но мне нужны, например, команда \tr для обозначения следа матрицы. Или я переопределяю LaTeX команды, которые работают не так, как мне хотелось бы. Типичный пример мнимая и вещественная часть комплексного числа \Im, \Re. В оригинале они выглядят не так, как мы привыкли. Кроме того, \Im еще используется и для обозначения образа линейного отображения. Подробнее описано внутри.
\usepackage{tex/commands}

% Потребуется для вставки картинки подписи
% \usepackage{graphicx}

% Пакет для титульника проекта
\usepackage{tex/titlepage}

\setToRo

\setTitle{\VAR{model.user.title}}

\setGroup{\VAR{model.user.group}}
\setStudent{\VAR{model.user.student}}
\setStudentDate{\VAR{model.user.date.strftime('%d.%m.%Y')}}
\setAdvisor{\VAR{model.user.advisor}}
\setAdvisorTitle{\VAR{model.user.advisor_title}}
\setAdvisorAffiliation{\VAR{model.user.advisor_affiliation}}
\setAdvisorDate{}
\setGrade{}
\setYear{\VAR{model.user.year}}


% С этого момента начинается текст документа
\begin{document}

% Эта команда создает титульную страницу
\makeTitlePage

% Здесь будет автоматически генерироваться содержание документа
\tableofcontents

\section{Назначение программы}
\subsection{Функциональное назначение программы}
\VAR{model.func_purpose}

\subsection{Эксплуатационное назначение программы}
\VAR{model.use_purpose}

\subsection{Состав функций}
\begin{itemize}
    \BLOCK{ for item in model.func_desc: }
    \item \textit{\VAR{item.name}} --- \VAR{item.text}
    \BLOCK{ endfor }
\end{itemize}

\section{Условия выполнения программы}
\subsection{Минимальный состав аппаратных средств}
\begin{itemize}
    \BLOCK{ for item in model.pc_cond: }
    \item \VAR{item}
    \BLOCK{ endfor }
\end{itemize}

\subsection{Минимальный состав программных средств}
\begin{itemize}
    \BLOCK{ for item in model.prog_cond: }
    \item \VAR{item}
    \BLOCK{ endfor }
\end{itemize}

\subsection{Требования к персоналу (пользователю)}
\begin{itemize}
    \BLOCK{ for item in model.user_cond: }
    \item \VAR{item}
    \BLOCK{ endfor }
\end{itemize}

\section{Выполнение программы}
\subsection{Загрузка и запуск программы}
\VAR{model.program_start}

\subsection{Выполнение программы}
\begin{itemize}
    \BLOCK{ for item in model.func_work: }
    \item \textit{\VAR{item.name}} --- \VAR{item.text}
    \BLOCK{ endfor }
\end{itemize}
    
\subsection{Завершение работы программы}
\VAR{model.program_end}
    
\section{Сообщения  оператору}
\BLOCK{ for msg in model.msgs: }
\subsection{\VAR{msg.name}}
\paragraph{Причина. } \VAR{msg.items[0]}
\paragraph{Действия программы. } \VAR{msg.items[1]}
\paragraph{Действия оператора. } \VAR{msg.items[2]}
\BLOCK{ endfor }

\end{document}
