% Это основная команда, с которой начинается любой \LaTeX-файл. Она отвечает за тип документа, с которым связаны основные правил оформления текста.
\documentclass[allcolors=black]{article}

% Здесь идет преамбула документа, тут пишутся команды, которые настраивают LaTeX окружение, подключаете внешние пакеты, определяете свои команды и окружения. В данном случае я это делаю в отдельных файлах, а тут подключаю эти файлы.

% Здесь я подключаю разные стилевые пакеты. Например возможности набирать особые символы или возможность компилировать русский текст. Подробное описание внутри.
\usepackage{tex/packages}

% Здесь я определяю разные окружения, например, теоремы, определения, замечания и так далее. У этих окружений разные стили оформления, кроме того, эти окружения могут быть нумерованными или нет. Все подробно объяснено внутри.
\usepackage{tex/environments}

% Здесь я определяю разные команды, которых нет в LaTeX, но мне нужны, например, команда \tr для обозначения следа матрицы. Или я переопределяю LaTeX команды, которые работают не так, как мне хотелось бы. Типичный пример мнимая и вещественная часть комплексного числа \Im, \Re. В оригинале они выглядят не так, как мы привыкли. Кроме того, \Im еще используется и для обозначения образа линейного отображения. Подробнее описано внутри.
\usepackage{tex/commands}

% Потребуется для вставки картинки подписи
% \usepackage{graphicx}

% Пакет для титульника проекта
\usepackage{tex/titlepage}

\setToPz

\setTitle{\VAR{model.user.title}}

\setGroup{\VAR{model.user.group}}
\setStudent{\VAR{model.user.student}}
\setStudentDate{\VAR{model.user.date.strftime('%d.%m.%Y')}}
\setAdvisor{\VAR{model.user.advisor}}
\setAdvisorTitle{\VAR{model.user.advisor_title}}
\setAdvisorAffiliation{\VAR{model.user.advisor_affiliation}}
\setAdvisorDate{}
\setGrade{}
\setYear{\VAR{model.user.year}}


% С этого момента начинается текст документа
\begin{document}

% Эта команда создает титульную страницу
\makeTitlePage

% Здесь будет автоматически генерироваться содержание документа
\tableofcontents

\section{Введение}
\VAR{model.intro}

\section{Назначение и область применения}
\subsection{Назначение программы}
\VAR{model.purpose}
    
\subsection{Характеристика области применения программы}
\VAR{model.space}

\section{Технические характеристики}
\subsection{Постановка задачи на разработку программы, описание применяемых математических методов и, при необходимости, описание допущений и ограничений, связанных с выбранным математическим материалом}
\VAR{model.task}

\subsection{Описание алгоритма программы с обоснованием выбора схемы алгоритма решения задачи}
\VAR{model.algo}

    
\subsection{Описание функционирования программы}
\VAR{model.func}

\subsection{Описание взаимодействия программы с другими программами}
\BLOCK{ for prog in model.communicate: }
\paragraph{\VAR{prog.name} }
\BLOCK{ for line in prog.items: }
\VAR{line}

\BLOCK{endfor}
\BLOCK{endfor}

\subsection{Описание и обоснование выбора состава технических средств}
\VAR{model.pc_cond}

\subsection{Описание и обоснование выбора состава программных средств}
\VAR{model.prog_cond}

\section{Ожидаемые технико-экономические показатели}
\VAR{model.wait}

\section{Обзор и сравнительный анализ аналогов}

\BLOCK{ for item in model.sources_review: }
\subsubsection*{\VAR{item.name}}
\VAR{item.text}
\BLOCK{endfor}

\subsection*{Итоги}
\VAR{model.sources_summary}

\end{document}
